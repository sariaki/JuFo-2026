\documentclass[11pt,a4paper,ngerman]{article}
\usepackage[utf8]{inputenc}
\usepackage[T1]{fontenc}

\usepackage{amsmath}
\usepackage{amsthm}
\usepackage{amstext}
\theoremstyle{definition}
\newtheorem{definition}{Definition}[section]
\theoremstyle{plain}
\newtheorem{example}{Beispiel}

% Times New Roman
\usepackage{newtxtext,newtxmath}
\usepackage[ngerman]{babel}

\usepackage{geometry}
\geometry{left=2.5cm, right=2.5cm, top=2.5cm, bottom=2cm}

\usepackage{setspace}
\linespread{1.25} % Word berechnet Zeilenabstände falsch...

\usepackage{graphicx}
\graphicspath{{img/}}

\usepackage{wrapfig}
\usepackage{caption}
\captionsetup[wrapfigure]{
  font=small,            
  labelfont=bf           
}
\captionsetup[wrapfigure]{}

\usepackage{listings}
\usepackage{algorithm}
\usepackage{algpseudocode}

\usepackage{biblatex}
\addbibresource{bib.bib}

% \usepackage{csquotes}
\usepackage{booktabs}  
% \usepackage{float} 

% Muss immer letztes Packet sein
\usepackage{hyperref}
\usepackage{cleveref}

% \title{\bf\Huge SArSO: Symbolische Ausführung resistenter Software-Obfuskation mittels opaker Prädikate unentscheidbarer stochastischer Funktionen}
\title{\bf\Huge SAroP: Symbolische Ausführung resistenter opake Prädikate mittels stochastischer Unentscheidbarkeit}
\date{}

\begin{document}

\begin{titlepage}
\centering
  \maketitle
  \vfill
  \begin{tabular}{ll}
    Teilnehmende: & Paul Baumgartner (18 J.) \\
    Erarbeitungsort: & Hildesheim \\
    Projektbetreuende: & Dr. Arndt Latußeck\\
    Fachgebiet: & Mathematik/Informatik \\
    Wettbewerbssparte: & Jugend forscht \\
    Bundesland: & Niedersachsen \\
    Wettbewerbsjahr: & 2026 \\
  \end{tabular}
\end{titlepage}

\thispagestyle{empty}
\tableofcontents

\newpage
\setcounter{page}{0}
\section{Projektüberblick}
%In diesem Projekt wird eine neue Softwareobfuskationsmethode präsentiert, welche sich als resistent gegenüber Symbolischer Ausführungsattacken erweist. ...
In diesem Projekt wird eine neue Klasse an opaken Prädikaten präsentiert; die Klasse stochastischer Prädikate, welche sich als resisten gegenüber symbolischen Ausführungsattacken erweist. ...
\newpage

\section{Einleitung}
% TODO: Vielleicht was auf Makro-Ebene sagen? Krieg, geopol. Spannungen, Cyber-Angriffe
\emph{Obfuskation} (lat. \emph{obfuscare}: verdunkeln) bezeichnet jede Transformation von Programmen zur Hinderung von sog. Reverse Engineering - der Analyse von Software zum Cracken, Verstehen oder Kopieren. Obfuskation kommt zum Einsatz in der Malwareentwicklung - um vor Detektion von sog. EDRS zu schützen, in der Industrie - um vor Kopien von Softwarefunktionen sowie vor Cracking zu schützen und im Militär - um dem Feind ein Verstädnis der eigenen Waffensysteme zu behindern. Da das Programm hierbei noch die Ursprüngliche Semantik beibehält kann jede Software mit genügend Zeit, Aufwand und Geld trotz Obfuskation verstanden werden. Der Sinn von Obfuskation ist also nicht die komplette Verhinderung von ''Reverse Engineering'', sondern vielmehr dieses wirtschaftlich unrentabel zu machen.

Diese Arbeit fokussiert sich auf eine Art der Obfuskation, der Kontrollflussobfuskation, und der meist verbreiteten Angriffsweise, um diese zu bekämpfen.

% TODO: Annahmen klären: Grundverständnis von Assembly, Kompilern etc.
\subsection{Beitrag dieser Arbeit}
% TODO: Klären, was in welchem Abschnitt gemacht wird.

\section{Theoretische Grundlagen}
\subsection{Obfuskation}
Collberg et al. \cite{collbergATaxonomyOfObfuscatingTransformations} definieren Obfuskation wie folgt:
\begin{definition}[Obfuskation]
Sei $P \xrightarrow{\mathcal{T}}P'$ eine Transformation eines \emph{Quellprogrammes} $P$ zu einem \emph{Zielprogramm} $P'$.
Eine solche Transformation ist eine Obfuskation, wenn das obfuskierte Programm $P'$ dasselbe beobachtbare Verhalten wie $P$ für den Endnutzer aufweist. 
\end{definition}

Per Definition sind somit Nebenwirkungen (z.B. Herunterladen von neuen Daten, Dateierzeugung etc.) erlaubt, solange sie nicht vom Nutzer erfahren werden. Die präsentierte Methode dieser Publikation nutzt diese Lockerung der Einschränkungen auf obfuskierende Transformationen stark aus, wie später ersichtlich sein wird.

Um eine Obfuskationsmethode zu evaluieren werden typischerweise die Metriken \emph{Stärke}, \emph{Resilienz}, \emph{Kosten} und \emph{Tarnung} verwendet \cite{collbergATaxonomyOfObfuscatingTransformations}. \label{par:metriken}
%Das Rückgängigmachen einer Obfuskation ist die \emph{Deobfuskation}.
%TODO: Arten der Obfuskation?

\subsection{Opake Prädikate}
Die folgenden Definitionen sind aus \cite{tofighi-shiraziDefeatingOpaquePredicates2019} sowie vom Pionierwerk \cite{collbergManufacturingCheapResilient1998} modifiziert übernommen. Es wird sich auf invariante opake Prädikate beschränkt. Ferner wird angenommen, dass alle Funktionen/Prädikatenterme evaluierbar sind.
\begin{definition}[Opake Prädikate]
Sei $\mathcal{O} : \Phi \rightarrow \{0, 1\}$ eine Abbildung einer Variable $\phi\in \Phi$ zu einem Prädikat. Das Prädikat $\mathcal{O}(\phi)$ ist opak, wenn für alle $\phi\in \Phi$ gilt, dass $\mathcal{O}(\phi)$ denselben Wert ($1$ oder $0$ bzw. wahr oder falsch) hat.
\end{definition}
In anderen Worten: Das Prädikat $\mathcal{O}(\phi)$ ist opak, wenn dessen Wert für alle möglichen Parameter \emph{a priori} bestimmt ist (also für den Programmierer bekannt ist) aber für ein Verständnis einer weiteren Person (ein Angreifer) \emph{a posteriori} (durch Beobachtung) zu bestimmen ist.

Diese Arbeit unterscheidet zwischen zwei Arten opaker Prädikate:
\begin{definition}
Sei $\mathcal{O} : \Phi \rightarrow \{0, 1\}$ ein opakes Prädikat.
$\mathcal{O}(\phi)$ ist vom Typ
\begin{enumerate} 
    \item $P^T$, wenn für alle $\phi \in \Phi$ gilt: $\mathcal{O}(\phi)=1\text{ bzw. } wahr$. 
    \item $P^F$, wenn für alle $\phi \in \Phi$ gilt: $\mathcal{O}(\phi)=0\text{ bzw. } falsch$. 
\end{enumerate}
\end{definition}
% TODO: Beispiele
% \begin{example}
% ...
% \end{example}
% \begin{wrapfigure}{r}{0.25\textwidth} \label{fig:overwatchGraph}
%     \includegraphics[
%         scale=1.25
%         % trim=0pt 700pt 1000pt 0pt
%         % clip
%     ]{overwatch_byfron_cropped}
%     \caption{Kontrollflussgraph einer Funktion mit opakem Prädikat. Abbildung aus der Disassembly des Spiels ''Overwatch'' mittels IDA entnommen.}
% \end{wrapfigure}
% \begin{wrapfigure}{r}{0.25\textwidth}
%     \includegraphics[
%         scale=0.25
%         % trim=0pt 700pt 1000pt 0pt
%         % clip
%     ]{pseudocode_cropped}
%     \caption{IDA Pseudocode von \cref{fig:overwatchGraph}.}
% \end{wrapfigure}
\begin{minipage}[t]{0.45\textwidth}
  \centering
  % \includegraphics[width=\linewidth]{overwatch_byfron_cropped}
  \includegraphics[ 
    scale=1.25
    % trim=0pt 700pt 1000pt 0pt
    % clip
    ]{overwatch_byfron_cropped}
  % \resizebox{0.9\textwidth}{!}{
  %   \newsavebox{\codes}
\newsavebox{\codel}
\newsavebox{\coder}

\newsavebox{\codes}
\begin{lrbox}{\codes}
\begin{lstlisting}[language={[x86masm]Assembler}]
sub_1819242E0 proc near
...
mov     dword ptr [rbp+0A60h], 8A25F401h
mov     eax, [rbp+0A60h]
test    al, 1
jz      short loc_1819243CD
\end{lstlisting}
\end{lrbox}

\newsavebox{\codel}
\begin{lrbox}{\codel}
\begin{lstlisting}[language={[x86masm]Assembler}]
...
retn
\end{lstlisting}
\end{lrbox}

\newsavebox{\coder}
\begin{lrbox}{\coder}
\begin{lstlisting}[language={[x86masm]Assembler}]
loc_1819243CD:
rep imul ebp, [rbx+rbp*8+0F2E668Eh], 841Fh
sub_1819242E0
\end{lstlisting}
\end{lrbox}

\begin{tikzpicture}[node distance = 2cm, auto]
    % Knoten
    \node [block] (start) {
        \usebox{\codes}
    };
    \coordinate [below=of start] (midpoint);
    \node [block, left=of midpoint] (left) {
        \usebox{\codel}
    };
    \node [block, right=of midpoint] (right) {
        \usebox{\coder}
    };
    

    % Kanten
    \path [line, color=green] (start) -- (left);
    \path [line, color=red] (start) -- (right);
\end{tikzpicture}
  % }
  \captionof{figure}{Kontrollflussgraph einer einfachen Funktion mit opakem Prädikat. Abbildung aus der Disassembly des Spiels ''Overwatch'' mittels IDA entnommen.} \label{fig:overwatchGraph}
\end{minipage}\hfill
\begin{minipage}[t]{0.5\textwidth}
  \centering
  % \resizebox{\columnwidth}{!}{\usebox{\idacode1}}
  \includegraphics[scale=0.2]{komplex_cropped}
  \captionof{figure}{Ausschnitt des Kontrollflussgraphen einer Funktion mit vielen opaken Prädikaten. Abbildung aus der Disassembly des Spiels ''Overwatch'' mittels IDA entnommen.} \label{fig:overwatchGraph2}
\end{minipage} \\

% TODO: erklären was das mit Obfuskation zu tun hat + auf Grafik eingehen
Opake Prädikate werden in der Softwareobfuskation eingesetzt, um ein Verständnis über den Kontrollfluss des Programms zu behindern. Damit opake Prädikate als Obfuskationsmethode \footnote{D.h, dass der wirkiche Pfad, welcher von einem opaken Prädikat verschleiert wird, nicht einfach erkannt werden kann} genutzt werden können, müssen sie wiederholt angewandt werden. Dadurch entsteht ein komplexerer Kontrollflussgraph und der Angreifer weiß folglich nicht, welche Basisblöcke zu analysieren sind. Die Stärke der opaken Prädikate ist hierbei abhängig von der Stärke ihres Terms/Ausdrucks. 

\begin{example}
Das Prädikat $\mathcal{O}(\phi)=-1977224191 \space \& \space 1 = 1$ aus \cref{fig:overwatchGraph}, wobei ''$\&$'' dem bitweisen ''und'' Operator entspricht, ist sehr einfach. Eine Berechnung genügt, um zu erkennen, dass das Prädikat immer wahr ist.
\end{example}


Mit zunehmender Komplexität der Prädikate und zunehmender Anzahl dieser, nimmt also auch die Obfuskationsstärke (Verwirrung und Unverständnis) beim Angreifer zu. 

%Mehrere Methoden zur Generierung solcher starken opaken Prädikate wurden bereits publiziert. Ein Überblick über diese sowie ihre Limitationen wird in \cref{sec:motivation} gegeben.
\subsection{Symbolische Ausführung} % Automatische Deobfuskationsattacken
%Symbolische Ausführung wird als Methode verwendet, um verschiedene Obfuskationsmethoden in Programmen anzugreifen. 
\begin{wrapfigure}{r}{0.3\textwidth}
  \centering
    \includegraphics[width=0.3\textwidth]{sym_exec}
  \caption{Konzeptuelles Framework zur Erkennung opaker Prädikate mit symbolischer Ausführung, Abb. aus \cite{xuManufacturingResilientBiOpaque2018} übernommen}
\end{wrapfigure}
Nach dem aktuellen Stand der Forschung scheint symbolische Ausführung der vielversprechendste Ansatz für die Bekämpfung opaker Prädikate zu sein.
Eine Symbolische Ausführungsmaschine besteht aus 2 Hauptkomponenten: ein zentrales symbolischen Ausführungsmodul und einem Modul zur Lösung von Einschränkungen (''constraint solver''). Es gibt zwei Arten symbolischer Ausführungsmodule: statisch und dynamisch (letzteres wird auch concolische symbolische Ausführung genannt). Concolische Ausführungsmaschinen wie BAP und Triton führen zunächst das Programm mit konkreten Werten aus und führen dann eine symbolsiche Analyse der generierten Befehlsspuren durch.
Statische symbolische Ausführungsmaschinen ''heben'' zuerst die Assembly-Anweisungen des Programmes in eine abstraktere Zwischensprache an und führt dann eine symbolsiche Ausführung dessen mit statischen Analyseansätzen durch. Diese Methode wird erfolgreich in Angr verwendet.

\section{Hintergrund und Motivation} \label{sec:motivation}
Der folgende Abschnitt soll einen Überblick über existierende Arten opaker Prädikate liefern und diese in Bezug auf ihre Qualität evaulieren. Hierfür werden die Faktoren aus \cref{par:metriken} verwendet.

\begin{table}[h]
    \centering
    \begin{tabular}{ccccc}
        \toprule
                & Stärke & Resilienz & Kosten & Tarnung \\
        \midrule
        Obfuscator-LLVM \cite{ieeespro2015-JunodRWM}            & X & trivial & sehr gering & hoch \\
        \emph{Bi-Opake} Prädikate \cite{xuManufacturingResilientBiOpaque2018}  & X & mittel & hoch & hoch \\
        \emph{Exceptions} \cite{linBranchObfuscationUsing2013}         & X & mittel & sehr hoch & gering \\
        NP-schwere Probleme \cite{}       & X & gering & mittel & mittel \\
        ungelöste Probleme \cite{wangLinearObfuscationCombat2011}        & X & hoch & sehr hoch & hoch \\
        \bottomrule
    \end{tabular}
    \caption{Vergleich existierende Arten opaker Prädikate.}
    \label{tab:vergleichderMethoden}
\end{table}
% Das Obfuscator-LLVM-Projekt \cite{ieeespro2015-JunodRWM} generiert opake Prädikate der Form $$y < 10 \vee  x \cdot (x + 1) \text{ mod } 2 == 0$$ wobei $x$ und $y$ symbolische Variablen sind, deren Wert statisch nicht bestimmbar ist. Ein Blick darauf zeigt, dass $\forall x:  x \cdot (x+1)$ gerade und somit durch $2$ teilbar ist. Die generierten opaken Prädikate sind somit auch vulnerabel gegenüber automatisierten symbolischen Ausführungsattacken. Stärke und Resilienz sind somit gering. Da nur wenige arithmetische Anweisungen zum Programm hinzugefügt werden sind Kosten gering und Tarnung hoch, da diese einem Angreifer nicht automatisch suspekt sind.

% Die sog. ''symbolischen'' Prädikate, wie sie in \cite{xuManufacturingResilientBiOpaque2018} vorgestellt wurden, versuchen eine solche Deobfuskation zu vermeiden. Hierfür nutzen diese Prädikate bewusste Konzepte, welche in den symbolischen Ausführungsmaschinen entweder gar nicht (wie z.B. Thread-Support) oder nur teilweise implementiert sind (wie z.B. Gleitkommazahlen).
% \begin{figure}[h]
%     \centering
%     \begin{lstlisting}[language=c++]
% int func(int symvar) {
%     int j = symvar;
%     float f = j/1000000.0;
%     if(f==0.1){
%         Bogus();
%     }
%     if(1024+f==1024&&f>0%%j==7){
%         Foo();
%     }
% }
% \end{lstlisting}
%     \caption{''symbollisches'' Prädikat mit Gleitkommazahlen, Abb. aus \cite{xuManufacturingResilientBiOpaque2018} übernommen.}
%     \label{fig:enter-label}
% \end{figure}
% % TODO: Stärke
% Zwar waren die Prädikate zur Zeit ihrer Veröffentlichung 2018 noch resilient gegenüber automatisierten Attacken - aufgrund ihrer Natur basiert deren Resilienz allerdings auf fehlenden Implementierungsfeinheiten der symbolischen Ausführungsmaschinen, nicht fundamentalen Limitationen derer. Jegliche Weiterentwicklung dieser Methode wird also von einer Anpassung der anderen Seite beantwortet - es handelt sich um ein spieltheoretisches Nullsummenspiel \cite{}.
% Je nach Methode variieren die Kosten stark von gering (z.B. bei Gleitkommazahlen) bis zu hoch (z.B. bei Threads).

% Ähnliche Probleme bezüglicher ihrer Resilienz haben \emph{exception}basierte opake Prädikate \cite{linBranchObfuscationUsing2013}. Zudem tragen diese sehr hohe Kosten

% TODO: Einfach Tabelle machen und dann losschreiben, warum wir *das* brauchen?

\begin{itemize}
    \item Obfuscator-LLVM \checkmark
    \item Bi-Opaque Prädikate \checkmark
    \item Linear Obfuscation to Combat Symbolic Execution
    \item Binary Code Side Effects \checkmark
    \item ''When Are Opaque Predicates Useful''
    \item ''Software obfuscation on a theoretical basis and its implementation,'' IEICE Trans. on Fundamentals of Electronics, Communications and Computer Sciences, 2003.
    \item (Collberg et al.) Tigress (dynamic opaque prediactes!!)
\end{itemize}

\section{Ansatz} 
\subsection{Angreifermodell}
Diese Arbeit geht aufgrund der Ähnlichkeit behandelter Thematik von einem ähnlichen Angreifermodell wie \cite{xuManufacturingResilientBiOpaque2018} aus.
Ein Angreifer hat direkten Zugriff auf das Programm und dessen Anweisungen. Es sei dem Angreifer hierbei nicht vorgegeben, wo und inwiefern das Programm obfuskiert ist. Der Angreifer kann das Programm nur statisch mittels symbolischer Ausführung sowie durch Untersuchung der Programmanweisungen analysieren und nicht ausführen. Solchen Situationen begegnet man z.B. bei Malware oder bei Software, welche für proprietäre unverfügbare Systeme geschrieben ist.
Eine Härtung gegen weitere (dynamische) Analysemethoden des Angreifers wird in \cref{sec:fazit} diskutiert.
\subsection{Algorithmus}
Während andere präsentierte Ansätze auf schweren Problemen beruhen, welche gelöst werden müsse, um einen Pfad auszuschließen, macht dieser Algorithmus dieses Ausschließen mit herkömmlichen Methoden unmöglich. Hierfür wird mit Wahrscheinlichkeit gearbeitet. Für jedes zu generierendes opakes Prädikat wird im Programm ein Pseudozufallsvariable $x$ generiert. Verschiedene Methoden hierfür werden in \cref{sec:pseudoZufall} gegeben. Wichtig ist, dass der Angreifer den Wert von $x$ nicht statisch bestimmen kann. Es wird angenommen, dass $x$ uniform verteilt ist. Diese Verteilung wird nun in eine andere (z.B. Normal-, Exponential-, Bernoulliverteilung) transformiert. 

Der Pseudocode hierfür ist in \cref{alg:stochasticOpaque} beschreiben.
\begin{algorithm}[!htb] 
    \caption{Generierung stochastischer opaker Prädikate}
    \label{alg:stochasticOpaque}
    \begin{algorithmic}[1]

    \State Erstelle Funktion $\mathit{TransformiereVerteilung}(x)$ im Modul, welche uniform verteilte Variable $x\in\space ]0; 1]$ normal verteilt. \\
    
    \Procedure{FügePrädikatEin}{$\mathit{Modul, T}$}
    \ForAll{Funktion in Modul}
        \State $BasisBlock \gets \Call{ZufälligerBasisBlock}{\mathit{Funktion}}$
        \State $\mathit{Entry} \gets \Call{ZufälligeAnweisung}{\mathit{BasisBlock}}$
        %\State $\mathit{LebendeVariable} \gets \Call{ZufälligeLebendeVariable}{\mathit{Entry}}$ \\
        \State $\mathit{ZufaelligerWert} \gets \Call{ErstelleFunktionsAufruf}{\mathit{UniformeVerteilung}, ]0; 1]}$ \\
        
        \State $\mathit{TransformierteVariable} \gets \Call{ErstelleFunktionsAufruf}{\mathit{TransformiereVerteilung, ZufaelligerWert}}$
        \State $\Call{ErstelleVergleich}{\mathit{TransformierteVariable}}$ \\
        
        \State $BB_{\mathrm{immer\_wahr}} \gets \Call{Splitblock}{\mathit{BasisBlock, Entry}}$
        \Comment{Basisblöcke für Prädikatenfälle generieren}
        \State $BB_{\mathrm{nie\_wahr}} \gets \Call{NeuerBlock}{\mathit{Funktion}}$ \\
    
        \State $\Call{ErstelleBedingteVerzweigung}{"\mathit{TransformierteVariable \geq T}", BB_{\mathrm{immer\_wahr}}, BB_{\mathrm{nie\_wahr}}}$ \\

        \State Fülle Basisblock $BB_{\mathrm{nie\_wahr}}$ mit zufälligen Anweisungen.
    \EndFor
    \EndProcedure
    \end{algorithmic}
\end{algorithm}
\subsection{Experimente}
\subsubsection{Optimale Wahrscheinlichkeitsverteilung}


\section{Implementierung}
\subsection{Entscheidungen}
LLVM am besten, weil viele Probleme gelöst von Experten; funktioniert auf Windows, Linux etc. (Kann zeit mit wichtigem verbringen); Sprachenunabhängig

rand() würde schlecht funktionieren, weil man SMT-solver konfigurieren könnte (auch hier hat man theoretisch gewonnen, weil sich Reverse-Engineer damit beschäftigen muss), sodass rand()=Konstante und somit alles determinierbar ist
\subsection{Wahrscheinlichkeitsverteilungsgenerierung}
\subsection{Generierung von Pseudozufallsvariablen} \label{sec:pseudoZufall}
\subsection{Generierung von ununterscheidbarem Füllcodes}

\section{Evaluierung} 
\subsection{Kriterien}
Kriterien nach Collberg et al. ...

\subsection{Vergleich mit existierenden Obfuskationsmethoden}

\section{Ergebnisdiskussion}

\section{Fazit und Ausblick} \label{sec:fazit}

\newpage
\pagenumbering{Alph}

%\section*{Quellen- und Literaturverzeichnis}
\printbibliography[title=Literaturverzeichnis]
\addcontentsline{toc}{section}{Literaturverzeichnis}
\listoffigures
\addcontentsline{toc}{section}{Abbildungsverzeichnis}

\end{document}